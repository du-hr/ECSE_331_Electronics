\documentclass[journal]{IEEEtran}
\usepackage[pdftex]{graphicx}
\usepackage[cmex10]{amsmath}
\usepackage{array}
\usepackage{mdwmath}
\usepackage{mdwtab}
\usepackage{eqparbox}
\usepackage{url}
\usepackage{pdfpages}
\usepackage{siunitx}

\begin{document}
\begin{titlepage}
\includepdf[page=-]{Cover.pdf}
\end{titlepage}
\title{MOSFETs and BJTs DC Characteristics}
\author{Chang Zhou, Siyu Wang, Haoran Du
\thanks{This work is the report of the laboratory section of course ECSE 331 offered at McGill University.}
\thanks{C. Zhou is with the Department of Electrical and Computer
Engineering, Faculty of Engineering, McGill University, H3A 0E9 Montreal, QC, Canada
(email: chang.zhou2@mail.mcgill.ca).}
\thanks{S. Wang is with the Department of Electrical and Computer
Engineering, Faculty of Engineering, McGill University, H3A 0E9 Montreal, QC, Canada
(email: siyu.wang5@mail.mcgill.ca).}
\thanks{H. Du is with the Department of Electrical and Computer
Engineering, Faculty of Engineering, McGill University, H3A 0E9 Montreal, QC, Canada
(email: haoran.du@mail.mcgill.ca).}}

\maketitle

\begin{abstract}
  The purpose of laboratory experiment was to explore the functions
  and characteristics of MOSFETs and BJTs. In the first part of this laboratory,
  the $I$-$V$ characteristics of the MOSFET was found and drawn for different gate voltages,
  then the transconductance $g_{m}$ of the circuit was found. In the second part,
  the behavior of the MOSFET was studied at various temperatures.
  The same experiment was also conducted on the BJT transistor.
\end{abstract}

\begin{IEEEkeywords}
  MOSFET, BJT, transistors
\end{IEEEkeywords}

\section{Introduction}

\IEEEPARstart{T}{he} goal of this laboratory was to test and explore the behavior of
 different transistors by drawing their $I$-$V$ diagrams using the NI Elvis-II test instrument.
 More specifically, the $I$-$V$ curve for the MOSFET and BJT transistors were drawn
 using the data taken with the NI Elvis instrument. A resistor network was designed
 to find the DC operating point of the transistors.
 Finally, the effect of temperature on the operation of the transistors was tested.
\section{Experiments Procedures and Result}
\subsection{MOSFET $I_{D}$-$V_{DS}$ Characteristics Using a Curve Tracer}
\begin{figure}[h]
  \centering
  \includegraphics[width=0.45\textwidth]{images/1.png}
  \caption{Circuit digram for I-V characteristic used in part A.}
  \label{fig-1}
\end{figure}
\par The above circuit shown in Fig. \ref{fig-1} was constructed to measure $V_{DS}$ and $I_{D}$
and trace the $I$-$V$ curve. This experiment was repeated multiple times with different
gate voltages $V_{g}$. A sawtooth waveform going from \SIrange{1}{10}{\volt} was applied.
\begin{figure}[h]
  \centering
  \includegraphics[width=0.45\textwidth]{images/1-2Vt.png}
  \caption{$V_{D}$ as a function of $V_{\text{in}}$ when $V_{g}=0$.}
  \label{fig-2}
\end{figure}
\par Next, in the same circuit, the gate voltage $V_{g}$ was increased to
\SIlist{3.1;3.6;4.1;4.6;5.1}{\volt}. For each case in the \SIrange{1}{5}{\volt}
gate voltage range, the $I$-$V$ curve was constructed using the data collected
with the Oscilloscope. Below are some of the results.
\begin{figure}[h]
\centering
\begin{minipage}{.25\textwidth}
  \centering
  \includegraphics[width=.8\linewidth]{images/1-31.png}
  \caption{$V_{g}=\SI{3.1}{\volt}$.}
\end{minipage}%
\begin{minipage}{.25\textwidth}
  \centering
  \includegraphics[width=.8\linewidth]{images/1-36.png}
  \caption{$V_{g}=\SI{3.6}{\volt}$.}
\end{minipage}
\end{figure}
\par Also, it is observed that when $V_{g}$ is greater than \SI{2}{\volt},
we start to see the behavior of the MOSFETs. When Vin is about to reach than
\SI{2}{\volt}, which is around the  threshold voltage according to the Manufacturers data sheet,
$I_{D}$ saturates.
\clearpage
\par To find the transconductance $g_{m}$, of our device the following formula:
\begin{equation*}
g_{m}=\frac{\Delta i_{D}}{\Delta v_{GS}}
\end{equation*}
\par Next, applying the circuit in Fig. \ref{fig-1}, we obtained the following results:
\begin{figure}[h]
\centering
\begin{minipage}{.25\textwidth}
  \centering
  \includegraphics[width=0.8\textwidth]{images/1-3dot1vg.png}
  \caption{Applying \SI{3.1}{\volt} $V_{GS}$.}
  \label{fig-vg1}
\end{minipage}%
\begin{minipage}{.25\textwidth}
  \centering
  \includegraphics[width=.8\linewidth]{images/1-3dot3vg.png}
  \caption{Applying \SI{3.3}{\volt} $V_{GS}$.}
  \label{fig-vg2}
\end{minipage}
\end{figure}
\par From Fig. \ref{fig-vg1} and Fig.\ref{fig-vg2}, hence,
\begin{equation*}
  \begin{split}
  g_{m}&=\frac{\Delta i_{D}}{\Delta v_{GS}}\\
  &=\frac{\SI{58}{\mA}-\SI{41}{\mA}}{\SI{3.3}{\V}-\SI{3.1}{\V}}\\
  &=\SI{0.085}{\siemens}
\end{split}
\end{equation*}
\par The equivalent small-signal model of MOSFET is presented in Fig. \ref{scan-1}:
\begin{figure}[h]
  \centering
  \includegraphics[width=0.45\textwidth]{images/scan-1.jpg}
  \caption{Equivalent small-signal model of MOSFET for low-frequency operation.}
  \label{scan-1}
\end{figure}
\par The hybrid-pi model was used in the above model, where we define
\begin{equation}
  r_{0}=\frac{V_{A}}{I_{D}}
\end{equation}
\subsection{MOSFET Temperature Effects}
\par The MOSFET’s $I_{D}$-$V_{DS}$ curve was measured using the circuit in Fig. \ref{fig-1}
at different temperatures. In theory, the current conducted at a higher temperature
a fixed $V_{DS}$ will be larger than that at a lower temperature. In the lab, the
following results are obtained at room temperature.
\begin{figure}[h]
  \centering
  \includegraphics[width=0.35\textwidth]{images/2-1.png}
  \caption{MOSFET’s $I_{D}$-$V_{y}$ at room temperature: \SI{31}{\celsius}.}
  \label{fig-5}
\end{figure}
\newpage
\begin{figure}[h]
  \centering
  \includegraphics[width=0.33\textwidth]{images/2-2.png}
  \caption{MOSFET’s $I_{D}$-$V_{y}$ at \SI{-18}{\celsius}.}
  \label{fig-6}
\end{figure}
\begin{figure}[h]
  \centering
  \includegraphics[width=0.33\textwidth]{images/2-3.png}
  \caption{MOSFET’s $I_{D}$-$V_{y}$ at \SI{47}{\celsius}.}
  \label{fig-7}
\end{figure}
\clearpage
\par By comparing the three graphs, we see that for \SI{31}{\celsius} and
\SI{47}{\celsius}, the values and the graph are very similar. For
\SI{-18}{\celsius}, $I_{D}$ is smaller at the same $V_{DS}$. It could be
explained by the fact that at lower temperatures, there are less
free electrons and holes available to conduct current. Similarly,
as the temperature increases, there would be more thermally generated holes
and free electrons, which increases its conductivity.
\subsection{BJT $I_{C}$-$V_{CE}$ Characteristics Using a Curve Tracer}\label{sec-3}
\begin{figure}[h]
  \centering
  \includegraphics[width=0.45\textwidth]{images/3-1.png}
  \caption{Circuit diagram to find $I_{C}$-$V_{CE}$ for the BJT.}
  \label{fig-8}
\end{figure}
\par To analyze the behavior of $I$-$V$ characteristic of the BJT, we
built the circuit in Fig. \ref{fig-8} and plot the curve for different values
of $V_{\text{in}}$, which is a sawtooth waveform that varies from \SIrange{1}{10}{\volt}.
We obtained the follwoing results:
\subsubsection{}
\begin{center}
    \begin{tabular}{ |c|c|c| }
        \hline
        $V_{CE}$ (unit: \si{\volt}) & $V_{Y}$ (unit: \si{\volt})&  $I_{C}$ (unit: \si{\ampere})\\
        \hline
        $0$ & $0$ & $0$ \\
        \hline
        $1$ & $0$ & $0$ \\
        \hline
        $2$ & $0$ & $0$ \\
        \hline
        \vdots & \vdots & \vdots \\
        \hline
        $8$ & $0$ & $0$ \\
        \hline
        $9$ & $0$ & $0$ \\
        \hline
        $10$ & $0$ & $0$ \\
        \hline
    \end{tabular}
\end{center}
\par When the base voltage $V_{B}$ is below \SI{650}{\mV}, the results are somewhat similar
to the table shown above and as shown in Fig. \ref{fig-3-2}, which is zero current flowing.
\begin{figure}[h]
  \centering
  \includegraphics[width=0.3\textwidth]{images/3-2.png}
  \caption{BJT is in switched off mode.}
  \label{fig-3-2}
\end{figure}
\subsubsection{}
\par When the base voltage $V_{B}$ is increased to \SI{700}{\mV}, we start to get the result
as shown in Fig. \ref{fig-3-3}.
\begin{figure}[h]
  \centering
  \includegraphics[width=0.3\textwidth]{images/3-3.png}
  \caption{Base voltage is increased to \SI{700}{\mV}.}
  \label{fig-3-3}
\end{figure}
\begin{center}
    \begin{tabular}{ |c|c|c| }
        \hline
        $V_{CE}$ (unit: \si{\volt}) & $V_{Y}$ (unit: \si{\volt})&  $I_{C}$ (unit: \si{\mA})\\
        \hline
        $0$ & $0.5$ & $5$ \\
        \hline
        $1$ & $1.5$ & $15.3$ \\
        \hline
        $2$ & $2$ & $15.3$ \\
        \hline
        \vdots & \vdots & \vdots \\
        \hline
        $5$ & $1.61$ & $16.1$ \\
        \hline
        $6$ & $1.69$ & $16.9$ \\
        \hline
        $7$ & $1.66$ & $16.6$ \\
        \hline
    \end{tabular}
\end{center}
\par As it can be seen from the graphs and tables, at low gate voltages, the value of $I_{C}$ is very small.
When the gate voltage is increased to around \SI{710}{\mV}, we see that the current starts to increase to around \SI{15}{\mA}.
\subsubsection{Computing the Transconductance of BJT}
\par Now, as required, the base voltage $V_{B}$ is set to be \SI{650}{\mV}. The result as
shown in Fig. \ref{fig-3-4}.
\begin{figure}[h]
  \centering
  \includegraphics[width=0.45\textwidth]{images/650mV.png}
  \caption{Base voltage is set to \SI{650}{\mV}.}
  \label{fig-3-4}
\end{figure}
\clearpage
\par From Fig. \ref{fig-3-4}, $I_{C}$ is measured to be  \SI{3}{\mA}. Next, we set base voltage $V_{B}$
to be \SI{700}{\mV}. The measuring result is shown in Fig. \ref{fig-3-5}.
\begin{figure}[h]
  \centering
  \includegraphics[width=0.45\textwidth]{images/700mV.png}
  \caption{Base voltage is set to \SI{700}{\mV}.}
  \label{fig-3-5}
\end{figure}
\par From Fig. \ref{fig-3-5}, $I_{C}$ is measured to be \SI{20}{\mA}. Hence,
\begin{equation*}
  \begin{split}
  g_{m}&=\frac{\Delta i_{c}}{\Delta v_{BE}}\\
  &=\frac{\SI{20}{\mA}-\SI{3}{\mA}}{\SI{700}{\mV}-\SI{650}{\mV}}\\
  &=\SI{0.34}{\siemens}
\end{split}
\end{equation*}
\subsubsection{Small signal model of BJT at low frequency}
\par The equivalent small-signal model of BJT is presented in Fig. \ref{scan-2}:
\begin{figure}[h]
  \centering
  \includegraphics[width=0.45\textwidth]{images/scan-2.jpg}
  \caption{Equivalent small-signal model of BJT.}
  \label{scan-2}
\end{figure}
\par The pi model was used in the above model since the emitter is
connected to the ground.
\begin{equation*}
  \begin{split}
    r_{0}&=\frac{V_{A}}{I_{c}}\\
    r_{\pi}&=\frac{V_{T}}{I_{B}}
  \end{split}
\end{equation*}
\subsection{BJT Temperature Effects}
\par As seen in subsection \ref{sec-3}, as we increase the temperature, the
current for a certain drain voltage increases. In theory, as
the temperature increases, there are more free holes and
electrons to carry the charges, thus the $I_{C}$-$V_{CE}$ curve would
shift up as the temperature increases. We obtained the following results:
\begin{figure}[h]
  \centering
  \includegraphics[width=0.35\textwidth]{images/4-1.png}
  \caption{BJT $V_{DS}$-$V_{Y}$ at \SI{22}{\celsius}.}
  \label{fig-9}
\end{figure}
\begin{figure}[h]
  \centering
  \includegraphics[width=0.35\textwidth]{images/4-2.png}
  \caption{BJT $V_{DS}$-$V_{Y}$ at \SI{45}{\celsius}.}
  \label{fig-10}
\end{figure}
\par By comparing Fig. \ref{fig-9} and Fig \ref{fig-10}, we see that for \SI{22}{\celsius}
and \SI{45}{\celsius}, at lower temperatures, there are less
free electrons and holes available to conduct current, which yields to a worse conductivity.
\section{Conclusion}
\par To conclude, in this lab, the experiments overall confirmed what was learned in class, various circuits were built and the usefulness of the MOSEFT and BJT transistors were demonstrated.
We saw in this lab that the conductivity of MOSFETs and BJTs are heavily affected by the temperature. Also, for different values of $V_{DS}$, the transistors behave differently. When $V_{DS}$ is more than the overdrive voltage, then its current  $I_{D}$ is saturated.
\end{document}
