\documentclass[journal]{IEEEtran}
\usepackage[pdftex]{graphicx}
\usepackage[cmex10]{amsmath}
\usepackage{array}
\usepackage{mdwmath}
\usepackage{mdwtab}
\usepackage{eqparbox}
\usepackage{url}
\usepackage{pdfpages}
\usepackage{siunitx}

\begin{document}
\begin{titlepage}
\includepdf[page=-]{Cover.pdf}
\end{titlepage}
\title{MOSFETs and BJTs DC Characteristics}
\author{Chang Zhou, Siyu Wang, Haoran Du
\thanks{This work is the report of the laboratory section of course ECSE 331 offered at McGill University.}
\thanks{C. Zhou is with the Department of Electrical and Computer
Engineering, Faculty of Engineering, McGill University, H3A 0E9 Montreal, QC, Canada
(email: chang.zhou2@mail.mcgill.ca).}
\thanks{S. Wang is with the Department of Electrical and Computer
Engineering, Faculty of Engineering, McGill University, H3A 0E9 Montreal, QC, Canada
(email: siyu.wang5@mail.mcgill.ca).}
\thanks{H. Du is with the Department of Electrical and Computer
Engineering, Faculty of Engineering, McGill University, H3A 0E9 Montreal, QC, Canada
(email: haoran.du@mail.mcgill.ca).}}

\maketitle

\begin{abstract}
  The purpose of laboratory experiment was to explore the functions
  and characteristics of MOSFETs and BJTs. In the first part of this laboratory,
  the $I$-$V$ characteristics of the MOSFET was found and drawn for different gate voltages,
  then the transconductance gm of the circuit was found. In the second part,
  the behavior of the MOSFET was studied at various temperatures.
  The same experiment was also conducted on the BJT transistor.
\end{abstract}

\begin{IEEEkeywords}
  MOSFET, BJT, transistors
\end{IEEEkeywords}

\section{Introduction}

\IEEEPARstart{T}{he} goal of this laboratory was to test and explore the behavior of
 different transistors by drawing their $I$-$V$ diagrams using the NI Elvis-II test instrument.
 More specifically, the $I$-$V$ curve for the MOSFET and BJT transistors were drawn
 using the data taken with the NI Elvis instrument. A resistor network was designed
 to find the DC operating point of the transistors.
 Finally, the effect of temperature on the operation of the transistors was tested.
\section{Experiments Procedures and Result}
\subsection{MOSFET $I_{D}$-$V_{DS}$ Characteristics Using a Curve Tracer}
\begin{figure}[h]
  \centering
  \includegraphics[width=0.45\textwidth]{images/1.png}
  \caption{Circuit digram for I-V characteristic used in part A.}
  \label{fig-1}
\end{figure}
\par The above circuit shown in Fig. \ref{fig-1} was constructed to measure $V_{DS}$ and $I_{D}$
and trace the $I$-$V$ curve. This experiment was repeated multiple times with different
gate voltages $V_{g}$. A sawtooth waveform going from \SIrange{1}{10}{\volt} was applied.
\begin{figure}[h]
  \centering
  \includegraphics[width=0.45\textwidth]{images/1-2Vt.png}
  \caption{$V_{D}$ as a function of $V_{\text{in}}$ when $V_{g}=0$.}
  \label{fig-2}
\end{figure}
\par Next, in the same circuit, the gate voltage $V_{g}$ was increased to
\SIlist{3.1;3.6;4.1;4.6;5.1}{\volt}. For each case in the \SIrange{1}{5}{\volt}
gate voltage range, the $I$-$V$ curve was constructed using the data collected
with the Oscilloscope. Below are some of the results.
\begin{figure}[h]
\centering
\begin{minipage}{.25\textwidth}
  \centering
  \includegraphics[width=.8\linewidth]{images/1-31.png}
  \caption{$V_{g}=\SI{3.1}{\volt}$.}
\end{minipage}%
\begin{minipage}{.25\textwidth}
  \centering
  \includegraphics[width=.8\linewidth]{images/1-36.png}
  \caption{$V_{g}=\SI{3.6}{\volt}$.}
\end{minipage}
\end{figure}
% \begin{figure}[h]
% \begin{minipage}{.25\textwidth}
%   \centering
%   \includegraphics[width=.8\linewidth]{images/1-41.png}
%   \caption{$V_{g}=\SI{4.1}{\volt}$.}
% \end{minipage}
% \begin{minipage}{.25\textwidth}
%   \centering
%   \includegraphics[width=.8\linewidth]{images/1-46.png}
%   \caption{$V_{g}=\SI{4.6}{\volt}$.}
% \end{minipage}
% % \begin{minipage}{.25\textwidth}
% %   \centering
% %   \includegraphics[width=.8\linewidth]{images/1-51.png}
% %   \caption{$V_{g}=\SI{5.1}{\volt}$.}
% % \end{minipage}
% \end{figure}
% \clearpage
\par Also, it is observed that when $V_{g}$ is greater than \SI{2}{\volt},
we start to see the behavior of the MOSFETs. When Vin is about to reach than
\SI{2}{\volt}, which is around the  threshold voltage according to the Manufacturers data sheet,
$I_{D}$ saturates.
\clearpage
\subsection{MOSFET Temperature Effects}
\par The MOSFET’s $I_{D}$-$V_{DS}$ curve was measured using the circuit in Fig. \ref{fig-1}
at different temperatures. In theory, the current conducted at a higher temperature
a fixed $V_{DS}$ will be larger than that at a lower temperature. In the lab, the
following results are obtained at room temperature.
\begin{figure}[h]
  \centering
  \includegraphics[width=0.45\textwidth]{images/2-1.png}
  \caption{MOSFET’s $I_{D}$-$V_{y}$ at room temperature: \SI{31}{\celsius}.}
  \label{fig-5}
\end{figure}
\par Using the circuit in  Fig. \ref{fig-1} with the MOSFET being at \SI{31}{\celsius},
we obtained the following data as written in table?.
\begin{figure}[h]
  \centering
  \includegraphics[width=0.45\textwidth]{images/2-2.png}
  \caption{MOSFET’s $I_{D}$-$V_{y}$ at \SI{-18}{\celsius}.}
  \label{fig-6}
\end{figure}
\begin{figure}[h]
  \centering
  \includegraphics[width=0.45\textwidth]{images/2-3.png}
  \caption{MOSFET’s $I_{D}$-$V_{y}$ at \SI{47}{\celsius}.}
  \label{fig-6}
\end{figure}
\par By comparing the three graphs, we see that for \SI{31}{\celsius} and
\SI{47}{\celsius}, the values and the graph are very similar. For
\SI{-18}{\celsius}, $I_{D}$ is smaller at the same $V_{DS}$. It could be
explained by the fact that at lower temperatures, there are less
free electrons and holes available to conduct current. Similarly,
as the temperature increases, there would be more thermally generated holes
and free electrons, which increases its conductivity.
\subsection{BJT $I_{C}$-$V_{CE}$ Characteristics Using a Curve Tracer}
\subsection{BJT Temperature Effects}
\section{Conclusion}

\end{document}
