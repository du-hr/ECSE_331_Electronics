%%%%%%%%%%%%%%%%%%%%%%%%%%%%%%%%%%%%%%%%%%%%%%%%%%%%%%%%%%%%%%%%%%%%%%%%%%%%%%%%
%2345678901234567890123456789012345678901234567890123456789012345678901234567890
%        1         2         3         4         5         6         7         8

\documentclass[letterpaper, 10 pt, conference]{ieeeconf}  % Comment this line out
% if you need a4paper

\IEEEoverridecommandlockouts                              % This command is only
% needed if you want to
% use the \thanks command
\overrideIEEEmargins
% See the \addtolength command later in the file to balance the column lengths
% on the last page of the document

\usepackage[utf8]{inputenc}
\usepackage[T1]{fontenc}
\usepackage{graphicx}
\usepackage{amsmath}
\usepackage{pdfpages}
\usepackage{fancyhdr}


\title{\Huge MOSFETs and BJTs DC Characteristics$^{*}$}

\author{Chang Zhou$^{1}$, Siyu Wang$^{2}$, Haoran Du$^{3}$% <-this % stops a space
\thanks{*This work is the report of the laboratory section of course ECSE 331 offered at McGill University.}% <-this % stops a space
\thanks{$^{1}$C. Zhou is with the Department of Electrical and Computer
Engineering, Faculty of Engineering, McGill University, Montreal, QC H3A 0E9 Canada,
(email: chang.zhou2@mail.mcgill.ca)}%
\thanks{$^{2}$S. Wang is with the Department of Electrical and Computer
Engineering, Faculty of Engineering, McGill University, Montreal, QC H3A 0E9 Canada,
(email: siyu.wang5@mail.mcgill.ca)}%
\thanks{$^{3}$H. Du is with the Department of Electrical and Computer
Engineering, Faculty of Engineering, McGill University, Montreal, QC H3A 0E9 Canada,
(email: haoran.du@mail.mcgill.ca)}%
}


\begin{document}

\includepdf[page=-]{Cover}
\maketitle
\thispagestyle{fancy}
\pagestyle{fancy}
\setcounter{page}{1}
\fancyhf{} % sets both header and footer to nothing
\renewcommand{\headrulewidth}{0pt} % clear decorative line
\rhead{\thepage}

%%%%%%%%%%%%%%%%%%%%%%%%%%%%%%%%%%%%%%%%%%%%%%%%%%%%%%%%%%%%%%%%%%%%%%%%%%%%%%%%
\begin{abstract}
The purpose of laboratory experiment was to explore the functions
and characteristics of MOSFETs and BJTs. In the first part of this laboratory,
the I-V characteristics of the MOSFET was found and drawn for different gate voltages,
then the transconductance gm of the circuit was found. In the second part,
the behavior of the MOSFET was studied at various temperatures.
The same experiment was also conducted on the BJT transistor.
\end{abstract}

%%%%%%%%%%%%%%%%%%%%%%%%%%%%%%%%%%%%%%%%%%%%%%%%%%%%%%%%%%%%%%%%%%%%%%%%%%%%%%%%
\section{INTRODUCTION}
\par The goal of this laboratory was to test and explore the behavior of
 different transistors by drawing their I-V diagrams using the NI Elvis-II test instrument.
 More specifically, the I-V curve for the MOSFET and BJT transistors were drawn
 using the data taken with the NI Elvis instrument. A resistor network was designed
 to find the DC operating point of the transistors.
 Finally, the effect of temperature on the operation of the transistors was tested.
\section{EXPERIMENTS PROCEDURES AND RESULT}
\subsection{MOSFET $I_{D}-V_{DS}$ Characteristics Using a Curve Tracer}
\subsection{MOSFET Temperature Effects}
\subsection{BJT $I_{C}-V_{CE}$ Characteristics Using a Curve Tracer}
\subsection{BJT Temperature Effects}




\section{CONCLUSIONS}

\end{document}
