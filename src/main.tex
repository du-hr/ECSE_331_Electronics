%%%%%%%%%%%%%%%%%%%%%%%%%%%%%%%%%%%%%%%%%%%%%%%%%%%%%%%%%%%%%%%%%%%%%%%%%%%%%%%%
%2345678901234567890123456789012345678901234567890123456789012345678901234567890
%        1         2         3         4         5         6         7         8

\documentclass[letterpaper, 10 pt, conference]{ieeeconf}  % Comment this line out
% if you need a4paper

\IEEEoverridecommandlockouts                              % This command is only
% needed if you want to
% use the \thanks command
\overrideIEEEmargins
% See the \addtolength command later in the file to balance the column lengths
% on the last page of the document

\usepackage[utf8]{inputenc}
\usepackage[T1]{fontenc}
\usepackage{graphicx}
\usepackage{amsmath}


\title{\LARGE \bf Silicon Diodes and Their Applications$^{*}$}

\author{Chang Zhou$^{1}$, Siyu Wang$^{2}$, Haoran Du$^{3}$% <-this % stops a space
\thanks{*This work is the report of the laboratory section of course ECSE 331 offered at McGill University.}% <-this % stops a space
\thanks{$^{1}$C. Zhou is with the Department of Electrical and Computer
Engineering, Faculty of Engineering, McGill University, Montreal, QC H3A 0E9 Canada.
{\tt\small (chang.zhou2@mail.mcgill.ca)}}%
\thanks{$^{2}$S. Wang is with the Department of Electrical and Computer
Engineering, Faculty of Engineering, McGill University, Montreal, QC H3A 0E9 Canada.
{\tt\small (siyu.wang5@mail.mcgill.ca)}}%
\thanks{$^{3}$H. Du is with the Department of Electrical and Computer
Engineering, Faculty of Engineering, McGill University, Montreal, QC H3A 0E9 Canada.
{\tt\small (haoran.du@mail.mcgill.ca)}}%
}


\begin{document}


    \maketitle
    \thispagestyle{empty}
    \pagestyle{empty}


    %%%%%%%%%%%%%%%%%%%%%%%%%%%%%%%%%%%%%%%%%%%%%%%%%%%%%%%%%%%%%%%%%%%%%%%%%%%%%%%%
    \begin{abstract}
        The purpose of this laboratory experiment was to explore the current-voltage properties of diodes. The main concepts we investigated in this lab are the rectification, zener diodes, AC to DC conversion, and forward and reverse bias.
    \end{abstract}


    %%%%%%%%%%%%%%%%%%%%%%%%%%%%%%%%%%%%%%%%%%%%%%%%%%%%%%%%%%%%%%%%%%%%%%%%%%%%%%%%
    \section{INTRODUCTION}
    Silicon diodes are two-terminal devices that are composed of the p-n conjunction. They are used for rectification. Standard diodes allow current in only one direction and do not conduct in reverse bias. However, zener diodes can operate in reverse-breakdown. This lab explores the properties of different diodes. We use an operational amplifier to keep track both the voltage across and the current through the diode under test.
\par In this report, we will present the result we obtained during the experiment and analysis them using knowlegde we learned from class.
    \section{EXPERIMENTS PROCEDURES AND RESULT}

    \subsection{Part 1: I-V Charactersitcs Using a Curve Tracer}
	\begin{figure}[h]
        \centering
        \includegraphics[width=0.5\textwidth]{images/Part1circuit.png}
        \caption{Circuit for i-v diode mearsurement in Part 1}
		\label{fig:1}
    \end{figure}
\par In the first part of the lab, before we construct the above circuit shown in Fig. \ref{fig:1} using the 1N4148 signal diode. We first verify that the voltage shown in scope B is 10 times greater than the current iD by replacing the diode with a 100 ohm resistor. The input signal is a triangle wave with 4 volt peak to peak voltage and zero offset. The experiment result is shown in Fig. \ref{fig:2}; the black one is Va and the grey one is Vb.Vb has an amplitude of 1.2645V (2.529 V peak-to-peak) and Va has an amplitude of 129.305 mV (258.61 mV peak-to-peak); by ohm's law the amplitude of the current through the resistor is 12.645  mA, which is about 10 times less than Vb. Thus, the circuit works properly.
	\begin{figure}[h]
        \centering
        \includegraphics[width=0.5\textwidth]{images/result1.png}
        \caption{Result for testing the op-amp; black is Va and grey is Vb}
		\label{fig:2}
    \end{figure}
\par Next, we replace the resistor with the 1N4148 signal diode, while keep the input signal the same as before. Fig. \ref{fig:3} shows the simulation and the experiment result.
	\begin{figure}[h]
        \centering
        \includegraphics[width=0.5\textwidth]{images/result2.png}
        \caption{Simulation result of the 1N4148 diode; black is Va and grey is Vb}
		\label{fig:3}
    \end{figure}
\par This simulation diagram by itself is not very valuable and nothing can really be generated of from this form; therefore, we plot the i-v diagram of this diode, and the result is shown below in Fig. \ref{fig:4}. By observing the graph, we find that the cut-in voltage is 0.5 V. Also, by using the formula slope = deltaI/deltaV, and the two pairs of data points (circled in red) from the i-v diagram in Fig. \ref{fig:5}, we calculate the slope at 0.7 V is 123.55 mA/V.
	\begin{figure}[h]
        \centering
        \includegraphics[width=0.5\textwidth]{images/result3.png}
        \caption{The i-v curve of the 1N4148 diode}
		\label{fig:4}
    \end{figure}
	\begin{figure}[h]
        \centering
        \includegraphics[width=0.5\textwidth]{images/result4.png}
        \caption{The data points from i-v curve of the 1N4148 diode; the first column represents the voltage in V; the second column represents the current in mA}
		\label{fig:5}
    \end{figure}
\par The maximum voltage difference between the actual i-v curve and that predicted by the piecewise linear model over a voltage range from -2 to +2 V is very small (0.1V). The piecewise linear approximation is a reliable model.
\par Then, we replace the 1N4148 signal diode with an 1N4005 diode, Fig. \ref{fig:6} shows the simulation and the experiment results. The results are similar.
	\begin{figure}[h]
        \centering
        \includegraphics[width=0.5\textwidth]{images/result5.png}
        \caption{Comparision of the i-v curves for 1N4005 diode and 1N4148 diode}
		\label{fig:6}
    \end{figure}
\par Finally, we reverse the 1N4148 diode, we observe that the behavior is also reversed because the triangle wave is symetric with respect to 0 V. Fig. \ref{fig:7} shows the i-v curve of the reversed 1N4148 diode.
	\begin{figure}[h]
        \centering
        \includegraphics[width=0.5\textwidth]{images/result6.png}
        \caption{The i-v curve of the reversed 1N4148 diode}
		\label{fig:7}
    \end{figure}
    \subsection{Part 2: Diode Temperature Effect}
	\begin{figure}[h]
        \centering
        \includegraphics[width=0.55\textwidth]{images/Part1circuit.png}
        \caption{Circuit for i-v diode mearsurement in Part 1}
		\label{fig:8}
    \end{figure}
    In part 2, we observe the behaviour of a diode under different temperatre conditions. The diode 1N4148 was tested first in room temperation, then in a lower temperature and a higher temperature by the above cuircuit in Fig. \ref{fig:8}. We keep track of the temperature by the hand-held thermal imager. 
\par By observing the results from in Fig. \ref{fig:9}, we find that the increase in the temperature will lower the cut in voltage of the diode. This is expected because as the overall energy in the electrons of semi-conductors increases, the amount of extra energy need to be provided to make the diode conduct decreases.\newline
\begin{figure}[h]
        \centering
        \includegraphics[width=0.55\textwidth]{images/result7.png}
        \caption{The i-v measurement for 1N4148 diode at different tempertures}
		\label{fig:9}
    \end{figure}


    \section{CONCLUSIONS}
    start here.

    % DO NOT NEED FROM HERE
    %  \addtolength{\textheight}{-12cm}
    % This command serves to balance the column lengths
    % on the last page of the document manually. It shortens
    % the textheight of the last page by a suitable amount.
    % This command does not take effect until the next page
    % so it should come on the page before the last. Make
    % sure that you do not shorten the textheight too much.

    %%%%%%%%%%%%%%%%%%%%%%%%%%%%%%%%%%%%%%%%%%%%%%%%%%%%%%%%%%%%%%%%%%%%%%%%%%%%%%%%



    %%%%%%%%%%%%%%%%%%%%%%%%%%%%%%%%%%%%%%%%%%%%%%%%%%%%%%%%%%%%%%%%%%%%%%%%%%%%%%%%



    %%%%%%%%%%%%%%%%%%%%%%%%%%%%%%%%%%%%%%%%%%%%%%%%%%%%%%%%%%%%%%%%%%%%%%%%%%%%%%%%

%    \section*{APPENDIX}
%
%    Appendixes should appear before the acknowledgment.
%
%    \section*{ACKNOWLEDGMENT}
%
%    The preferred spelling of the word ``acknowledgment'' in America is without an ``e'' after the ``g''. Avoid the stilted expression, ``One of us (R. B. G.) thanks . . .'' Instead, try ``R. B. G. thanks''. Put sponsor acknowledgments in the unnumbered footnote on the first page.
%
%
%
%    %%%%%%%%%%%%%%%%%%%%%%%%%%%%%%%%%%%%%%%%%%%%%%%%%%%%%%%%%%%%%%%%%%%%%%%%%%%%%%%%
%
%    References are important to the reader; therefore, each citation must be complete and correct. If at all possible, references should be commonly available publications.


%    \begin{thebibliography}{99}
%
%        \bibitem{c1} G. O. Young, ``Synthetic structure of industrial plastics (Book style with paper title and editor),'' in Plastics, 2nd ed. vol. 3, J. Peters, Ed. New York: McGraw-Hill, 1964, pp. 15--64.
%        \bibitem{c2} W.-K. Chen, Linear Networks and Systems (Book style). Belmont, CA: Wadsworth, 1993, pp. 123--135.
%        \bibitem{c3} H. Poor, An Introduction to Signal Detection and Estimation. New York: Springer-Verlag, 1985, ch. 4.
%        \bibitem{c4} B. Smith, ``An approach to graphs of linear forms (Unpublished work style),'' unpublished.
%        \bibitem{c5} E. H. Miller, ``A note on reflector arrays (Periodical styleÑAccepted for publication),'' IEEE Trans. Antennas Propagat., to be publised.
%        \bibitem{c6} J. Wang, ``Fundamentals of erbium-doped fiber amplifiers arrays (Periodical styleÑSubmitted for publication),'' IEEE J. Quantum Electron., submitted for publication.
%        \bibitem{c7} C. J. Kaufman, Rocky Mountain Research Lab., Boulder, CO, private communication, May 1995.
%        \bibitem{c8} Y. Yorozu, M. Hirano, K. Oka, and Y. Tagawa, ``Electron spectroscopy studies on magneto-optical media and plastic substrate interfaces(Translation Journals style),'' IEEE Transl. J. Magn.Jpn., vol. 2, Aug. 1987, pp. 740--741 [Dig. 9th Annu. Conf. Magnetics Japan, 1982, p. 301].
%        \bibitem{c9} M. Young, The Techincal Writers Handbook. Mill Valley, CA: University Science, 1989.
%        \bibitem{c10} J. U. Duncombe, ``Infrared navigationÑPart I: An assessment of feasibility (Periodical style),'' IEEE Trans. Electron Devices, vol. ED-11, pp. 34--39, Jan. 1959.
%        \bibitem{c11} S. Chen, B. Mulgrew, and P. M. Grant, ``A clustering technique for digital communications channel equalization using radial basis function networks,'' IEEE Trans. Neural Networks, vol. 4, pp. 570--578, July 1993.
%        \bibitem{c12} R. W. Lucky, ``Automatic equalization for digital communication,'' Bell Syst. Tech. J., vol. 44, no. 4, pp. 547--588, Apr. 1965.
%        \bibitem{c13} S. P. Bingulac, ``On the compatibility of adaptive controllers (Published Conference Proceedings style),'' in Proc. 4th Annu. Allerton Conf. Circuits and Systems Theory, New York, 1994, pp. 8--16.
%        \bibitem{c14} G. R. Faulhaber, ``Design of service systems with priority reservation,'' in Conf. Rec. 1995 IEEE Int. Conf. Communications, pp. 3--8.
%        \bibitem{c15} W. D. Doyle, ``Magnetization reversal in films with biaxial anisotropy,'' in 1987 Proc. INTERMAG Conf., pp. 2.2-1--2.2-6.
%        \bibitem{c16} G. W. Juette and L. E. Zeffanella, ``Radio noise currents n short sections on bundle conductors (Presented Conference Paper style),'' presented at the IEEE Summer power Meeting, Dallas, TX, June 22--27, 1990, Paper 90 SM 690-0 PWRS.
%        \bibitem{c17} J. G. Kreifeldt, ``An analysis of surface-detected EMG as an amplitude-modulated noise,'' presented at the 1989 Int. Conf. Medicine and Biological Engineering, Chicago, IL.
%        \bibitem{c18} J. Williams, ``Narrow-band analyzer (Thesis or Dissertation style),'' Ph.D. dissertation, Dept. Elect. Eng., Harvard Univ., Cambridge, MA, 1993.
%        \bibitem{c19} N. Kawasaki, ``Parametric study of thermal and chemical nonequilibrium nozzle flow,'' M.S. thesis, Dept. Electron. Eng., Osaka Univ., Osaka, Japan, 1993.
%        \bibitem{c20} J. P. Wilkinson, ``Nonlinear resonant circuit devices (Patent style),'' U.S. Patent 3 624 12, July 16, 1990.
%
%
%    \end{thebibliography}


\end{document}

