%%%%%%%%%%%%%%%%%%%%%%%%%%%%%%%%%%%%%%%%%%%%%%%%%%%%%%%%%%%%%%%%%%%%%%%%%%%%%%%%
%2345678901234567890123456789012345678901234567890123456789012345678901234567890
%        1         2         3         4         5         6         7         8

\documentclass[letterpaper, 10 pt, conference]{ieeeconf}  % Comment this line out
% if you need a4paper

\IEEEoverridecommandlockouts                              % This command is only
% needed if you want to
% use the \thanks command
\overrideIEEEmargins
% See the \addtolength command later in the file to balance the column lengths
% on the last page of the document

\usepackage[utf8]{inputenc}
\usepackage[T1]{fontenc}
\usepackage{graphicx}
\usepackage{amsmath}


\title{\LARGE \bf Silicon Diodes and Their Applications$^{*}$}

\author{Chang Zhou$^{1}$, Siyu Wang$^{2}$, Haoran Du$^{3}$% <-this % stops a space
\thanks{*This work is the report of the laboratory section of course ECSE 331 offered at McGill University.}% <-this % stops a space
\thanks{$^{1}$C. Zhou is with the Department of Electrical and Computer
Engineering, Faculty of Engineering, McGill University, Montreal, QC H3A 0E9 Canada.
{\tt\small (chang.zhou2@mail.mcgill.ca)}}%
\thanks{$^{2}$S. Wang is with the Department of Electrical and Computer
Engineering, Faculty of Engineering, McGill University, Montreal, QC H3A 0E9 Canada.
{\tt\small (siyu.wang5@mail.mcgill.ca)}}%
\thanks{$^{3}$H. Du is with the Department of Electrical and Computer
Engineering, Faculty of Engineering, McGill University, Montreal, QC H3A 0E9 Canada.
{\tt\small (haoran.du@mail.mcgill.ca)}}%
}


\begin{document}


    \maketitle
    \thispagestyle{empty}
    \pagestyle{empty}


    %%%%%%%%%%%%%%%%%%%%%%%%%%%%%%%%%%%%%%%%%%%%%%%%%%%%%%%%%%%%%%%%%%%%%%%%%%%%%%%%
    \begin{abstract}
        Write abstract here.
    \end{abstract}


    %%%%%%%%%%%%%%%%%%%%%%%%%%%%%%%%%%%%%%%%%%%%%%%%%%%%%%%%%%%%%%%%%%%%%%%%%%%%%%%%
    \section{INTRODUCTION}
    Write introduction here.
    \section{EXPERIMENTS PROCEDURES AND RESULT}

    \subsection{Part 1: I-V Charactersitcs Using a Curve Tracer}
    Start here.\newline
    Example 1.
    \begin{itemize}
        \item Item.
        \item Item.
        \item Item.
        \item Item.
    \end{itemize}
     Example 2.
    \begin{enumerate}
        \item Item.
        \item Item.
        \item Item.
        \item Item.
        \item Item.
    \end{enumerate}
    Example 3.
    \begin{flalign}
        \alpha + \beta = \chi
    \end{flalign}
    Example 4.
    \begin{figure}[h]
        \centering
        \includegraphics[width=0.25\textwidth]{images/download.png}
        \caption{example: McGill Logo}
    \end{figure}
    \begin{table}[h]
        \caption{An Example of a Table}
        \label{table_example}
        \begin{center}
            \begin{tabular}{|c||c|}
                \hline
                One & Two\\
                \hline
                Three & Four\\
                \hline
            \end{tabular}
        \end{center}
    \end{table}
    \subsection{Part 2: Diode Temperature Effect}
    Start here.\newline
    example: $x_{1}+x^{'} = \sum_{i=1}^N{K^{2}+1}$
    \subsection{Part 3: Zender Diodes}
    Start here.
    \subsection{Part 4; Rectifiers}
    Start here.
    \subsection{Part 5: Voltage Rugualtion Using Zender Diode}
    Start here.
    \subsection{Part 6: Limiter Circuit Using Diodes}
    Start here.

    \section{CONCLUSIONS}
    start here.

    % DO NOT NEED FROM HERE
    %  \addtolength{\textheight}{-12cm}
    % This command serves to balance the column lengths
    % on the last page of the document manually. It shortens
    % the textheight of the last page by a suitable amount.
    % This command does not take effect until the next page
    % so it should come on the page before the last. Make
    % sure that you do not shorten the textheight too much.

    %%%%%%%%%%%%%%%%%%%%%%%%%%%%%%%%%%%%%%%%%%%%%%%%%%%%%%%%%%%%%%%%%%%%%%%%%%%%%%%%



    %%%%%%%%%%%%%%%%%%%%%%%%%%%%%%%%%%%%%%%%%%%%%%%%%%%%%%%%%%%%%%%%%%%%%%%%%%%%%%%%



    %%%%%%%%%%%%%%%%%%%%%%%%%%%%%%%%%%%%%%%%%%%%%%%%%%%%%%%%%%%%%%%%%%%%%%%%%%%%%%%%

%    \section*{APPENDIX}
%
%    Appendixes should appear before the acknowledgment.
%
%    \section*{ACKNOWLEDGMENT}
%
%    The preferred spelling of the word ``acknowledgment'' in America is without an ``e'' after the ``g''. Avoid the stilted expression, ``One of us (R. B. G.) thanks . . .'' Instead, try ``R. B. G. thanks''. Put sponsor acknowledgments in the unnumbered footnote on the first page.
%
%
%
%    %%%%%%%%%%%%%%%%%%%%%%%%%%%%%%%%%%%%%%%%%%%%%%%%%%%%%%%%%%%%%%%%%%%%%%%%%%%%%%%%
%
%    References are important to the reader; therefore, each citation must be complete and correct. If at all possible, references should be commonly available publications.


%    \begin{thebibliography}{99}
%
%        \bibitem{c1} G. O. Young, ``Synthetic structure of industrial plastics (Book style with paper title and editor),'' in Plastics, 2nd ed. vol. 3, J. Peters, Ed. New York: McGraw-Hill, 1964, pp. 15--64.
%        \bibitem{c2} W.-K. Chen, Linear Networks and Systems (Book style). Belmont, CA: Wadsworth, 1993, pp. 123--135.
%        \bibitem{c3} H. Poor, An Introduction to Signal Detection and Estimation. New York: Springer-Verlag, 1985, ch. 4.
%        \bibitem{c4} B. Smith, ``An approach to graphs of linear forms (Unpublished work style),'' unpublished.
%        \bibitem{c5} E. H. Miller, ``A note on reflector arrays (Periodical styleÑAccepted for publication),'' IEEE Trans. Antennas Propagat., to be publised.
%        \bibitem{c6} J. Wang, ``Fundamentals of erbium-doped fiber amplifiers arrays (Periodical styleÑSubmitted for publication),'' IEEE J. Quantum Electron., submitted for publication.
%        \bibitem{c7} C. J. Kaufman, Rocky Mountain Research Lab., Boulder, CO, private communication, May 1995.
%        \bibitem{c8} Y. Yorozu, M. Hirano, K. Oka, and Y. Tagawa, ``Electron spectroscopy studies on magneto-optical media and plastic substrate interfaces(Translation Journals style),'' IEEE Transl. J. Magn.Jpn., vol. 2, Aug. 1987, pp. 740--741 [Dig. 9th Annu. Conf. Magnetics Japan, 1982, p. 301].
%        \bibitem{c9} M. Young, The Techincal Writers Handbook. Mill Valley, CA: University Science, 1989.
%        \bibitem{c10} J. U. Duncombe, ``Infrared navigationÑPart I: An assessment of feasibility (Periodical style),'' IEEE Trans. Electron Devices, vol. ED-11, pp. 34--39, Jan. 1959.
%        \bibitem{c11} S. Chen, B. Mulgrew, and P. M. Grant, ``A clustering technique for digital communications channel equalization using radial basis function networks,'' IEEE Trans. Neural Networks, vol. 4, pp. 570--578, July 1993.
%        \bibitem{c12} R. W. Lucky, ``Automatic equalization for digital communication,'' Bell Syst. Tech. J., vol. 44, no. 4, pp. 547--588, Apr. 1965.
%        \bibitem{c13} S. P. Bingulac, ``On the compatibility of adaptive controllers (Published Conference Proceedings style),'' in Proc. 4th Annu. Allerton Conf. Circuits and Systems Theory, New York, 1994, pp. 8--16.
%        \bibitem{c14} G. R. Faulhaber, ``Design of service systems with priority reservation,'' in Conf. Rec. 1995 IEEE Int. Conf. Communications, pp. 3--8.
%        \bibitem{c15} W. D. Doyle, ``Magnetization reversal in films with biaxial anisotropy,'' in 1987 Proc. INTERMAG Conf., pp. 2.2-1--2.2-6.
%        \bibitem{c16} G. W. Juette and L. E. Zeffanella, ``Radio noise currents n short sections on bundle conductors (Presented Conference Paper style),'' presented at the IEEE Summer power Meeting, Dallas, TX, June 22--27, 1990, Paper 90 SM 690-0 PWRS.
%        \bibitem{c17} J. G. Kreifeldt, ``An analysis of surface-detected EMG as an amplitude-modulated noise,'' presented at the 1989 Int. Conf. Medicine and Biological Engineering, Chicago, IL.
%        \bibitem{c18} J. Williams, ``Narrow-band analyzer (Thesis or Dissertation style),'' Ph.D. dissertation, Dept. Elect. Eng., Harvard Univ., Cambridge, MA, 1993.
%        \bibitem{c19} N. Kawasaki, ``Parametric study of thermal and chemical nonequilibrium nozzle flow,'' M.S. thesis, Dept. Electron. Eng., Osaka Univ., Osaka, Japan, 1993.
%        \bibitem{c20} J. P. Wilkinson, ``Nonlinear resonant circuit devices (Patent style),'' U.S. Patent 3 624 12, July 16, 1990.
%
%
%    \end{thebibliography}


\end{document}
