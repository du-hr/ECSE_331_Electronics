%%%%%%%%%%%%%%%%%%%%%%%%%%%%%%%%%%%%%%%%%%%%%%%%%%%%%%%%%%%%%%%%%%%%%%%%%%%%%%%%
%2345678901234567890123456789012345678901234567890123456789012345678901234567890
%        1         2         3         4         5         6         7         8

\documentclass[letterpaper, 10 pt, conference]{ieeeconf}  % Comment this line out
% if you need a4paper

\IEEEoverridecommandlockouts                              % This command is only
% needed if you want to
% use the \thanks command
\overrideIEEEmargins
% See the \addtolength command later in the file to balance the column lengths
% on the last page of the document

\usepackage[utf8]{inputenc}
\usepackage[T1]{fontenc}
\usepackage{graphicx}
\usepackage{amsmath}


\title{\LARGE \bf Silicon Diodes and Their Applications$^{*}$}

\author{Chang Zhou$^{1}$, Siyu Wang$^{2}$, Haoran Du$^{3}$% <-this % stops a space
\thanks{*This work is the report of the laboratory section of course ECSE 331 offered at McGill University.}% <-this % stops a space
\thanks{$^{1}$C. Zhou is with the Department of Electrical and Computer
Engineering, Faculty of Engineering, McGill University, Montreal, QC H3A 0E9 Canada.
{\tt\small (chang.zhou2@mail.mcgill.ca)}}%
\thanks{$^{2}$S. Wang is with the Department of Electrical and Computer
Engineering, Faculty of Engineering, McGill University, Montreal, QC H3A 0E9 Canada.
{\tt\small (siyu.wang5@mail.mcgill.ca)}}%
\thanks{$^{3}$H. Du is with the Department of Electrical and Computer
Engineering, Faculty of Engineering, McGill University, Montreal, QC H3A 0E9 Canada.
{\tt\small (haoran.du@mail.mcgill.ca)}}%
}


\begin{document}


\maketitle
\thispagestyle{empty}
\pagestyle{empty}


%%%%%%%%%%%%%%%%%%%%%%%%%%%%%%%%%%%%%%%%%%%%%%%%%%%%%%%%%%%%%%%%%%%%%%%%%%%%%%%%
\begin{abstract}
  The purpose of this laboratory experiment was to explore the current-voltage properties of diodes. The main concepts we investigated in this lab are the rectification, zener diodes, AC to DC conversion, and forward and reverse bias.
\end{abstract}


%%%%%%%%%%%%%%%%%%%%%%%%%%%%%%%%%%%%%%%%%%%%%%%%%%%%%%%%%%%%%%%%%%%%%%%%%%%%%%%%
\section{INTRODUCTION}
Silicon diodes are two-terminal devices that are composed of the p-n conjunction. They are used for rectification. Standard diodes allow current in only one direction and do not conduct in reverse bias. However, zener diodes can operate in reverse-breakdown. This lab explores the properties of different diodes. We use an operational amplifier to keep track both the voltage across and the current through the diode under test.
\par In this report, we will present the result we obtained during the experiment and analysis them using knowledge we learned from class.
\section{EXPERIMENTS PROCEDURES AND RESULT}
\subsection{Part 1: I-V Characteristics Using a Curve Tracer}
\begin{figure}[h]
  \centering
  \includegraphics[width=0.5\textwidth]{images/Part1circuit.png}
  \caption{Circuit for i-v diode measurement in Part 1}
  \label{fig:1}
\end{figure}
\par In the first part of the lab, before we construct the above circuit shown in Fig. \ref{fig:1} using the 1N4148 signal diode. We first verify that the voltage shown in scope B is 10 times greater than the current $i_{D}$ by replacing the diode with a 100 $\Omega$ resistor. The input signal is a triangle wave with 4 $V$ peak to peak voltage and zero offset. The experiment result is shown in Fig. \ref{fig:2}; the black one is $V_{a}$ and the grey one is $V_{b}$. $V_{b}$ has an amplitude of 1.2645 $V$ and $V_{a}$ has an amplitude of 129.305 $mV$ (258.61 $mV$ peak-to-peak); by Ohm's law the amplitude of the current through the resistor is 12.645  $mA$, which is about 10 times less than $V_{b}$. Thus, the circuit works properly.
\begin{figure}[h]
  \centering
  \includegraphics[width=0.32\textwidth]{images/result1.png}
  \caption{Result for testing the op-amp; black is Va and grey is Vb}
  \label{fig:2}
\end{figure}
\par Next, we replace the resistor with the 1N4148 signal diode, while keep the input signal the same as before. Fig. \ref{fig:3} shows the simulation and the experiment result.
\begin{figure}[h]
  \centering
  \includegraphics[width=0.32\textwidth]{images/result2.png}
  \caption{Simulation result of the 1N4148 diode; black is Va and grey is Vb}
  \label{fig:3}
\end{figure}
\clearpage
\begin{figure}[h]
  \centering
  \includegraphics[width=0.4\textwidth]{images/result3.png}
  \caption{The i-v curve of the 1N4148 diode}
  \label{fig:4}
\end{figure}
\par This simulation diagram by itself is not very valuable and nothing can really be generated of from this form; therefore, we plot the i-v diagram of this diode, and the result is shown above in Fig. \ref{fig:4}. By observing the graph, we find that the cut-in voltage is 0.5 V. Also, by using the formula slope = deltaI/deltaV, and the two pairs of data points (circled in red) from the i-v diagram in Fig. \ref{fig:5}, we calculate the slope at 0.7 V is 123.55 mA/V.
\begin{figure}[h]
  \centering
  \includegraphics[width=0.3\textwidth]{images/result4.png}
  \caption{The data points from i-v curve of the 1N4148 diode; the first column represents the voltage in V; the second column represents the current in mA}
  \label{fig:5}
\end{figure}
\par The maximum voltage difference between the actual i-v curve and that predicted by the piecewise linear model over a voltage range from -2 to +2 V is very small (0.1V). The piecewise linear approximation is a reliable model.
\par Then, we replace the 1N4148 signal diode with an 1N4005 diode, Fig. \ref{fig:6} shows the simulation and the experiment results. The results are similar.
\begin{figure}[h]
  \centering
  \includegraphics[width=0.5\textwidth]{images/result5.png}
  \caption{Comparision of the i-v curves for 1N4005 diode and 1N4148 diode}
  \label{fig:6}
\end{figure}
\newpage
\par Finally, we reverse the 1N4148 diode, we observe that the behavior is also reversed because the triangle wave is symmetric with respect to 0 V. Fig. \ref{fig:7} shows the i-v curve of the reversed 1N4148 diode.
\begin{figure}[h]
  \centering
  \includegraphics[width=0.5\textwidth]{images/result6.png}
  \caption{The i-v curve of the reversed 1N4148 diode}
  \label{fig:7}
\end{figure}
\clearpage
\subsection{Part 2: Diode Temperature Effect}
\begin{figure}[ht]
  \centering
  \includegraphics[width=0.5\textwidth]{images/Part1circuit.png}
  \caption{Circuit for i-v diode measurement in Part 1}
  \label{fig:8}
\end{figure}
In part 2, we observe the behavior of a diode under different temperature conditions. The diode 1N4148 was tested first in room temperature, then in a lower temperature and a higher temperature by the above circuit in Fig. \ref{fig:8}. We keep track of the temperature by the hand-held thermal imager.
\par By observing the results from in Fig. \ref{fig:9}, we find that the increase in the temperature will lower the cut in voltage of the diode. This is expected because as the overall energy in the electrons of semi-conductors increases, the amount of extra energy need to be provided to make the diode conduct decreases.\newline
\begin{figure}[ht]
  \centering
  \includegraphics[width=0.5\textwidth]{images/result7.png}
  \caption{The i-v measurement for 1N4148 diode at different temperatures}
  \label{fig:9}
\end{figure}
\newpage
\subsection{Part 3: Zener Diodes}
\begin{figure}[ht]
  \centering
  \includegraphics[width=0.3\textwidth]{images/zener_circuit.png}
  \caption{Circuit for capturing the i-v characteristic of a 3.3 $V$ Zener diode.}
  \label{fig:zener_1}
\end{figure}
Fig. \ref{fig:zener_1} shows the circuit that was applied to conclude the characteristics of
a Zener diode. A sawtooth input signal and varied amplitude was applied to the circuit. The behavior
of the Zener diode is shown below in Fig. \ref{fig:zener_2} and Fig. \ref{fig:zener_3}.
\begin{figure}[h]
  \centering
  \includegraphics[width=0.3\textwidth]{images/zener_simu.png}
  \caption{Simulation results of the Zener diode.}
  \label{fig:zener_2}
\end{figure}
\begin{figure}[h]
  \centering
  \includegraphics[width=0.3\textwidth]{images/zener_iv.png}
  \caption{I-V curve of the Zener diode.}
  \label{fig:zener_3}
\end{figure}
\clearpage
This I-V curve clearly shows that the Zener diode is operating in the breakdown region.
The Zener diode starts conducting large current when the voltage across it is below
a certain value, in this case that seems to be around -7.5 $V$.
Thus, the Zener diode breakdown happens around a voltage of -7.5 $V$.
\par The AC resistance inside the breakdown region is $\frac{1}{slope}$ of the breakdown region,
estimated from the graph to be about 15.3 $\Omega$.
\subsection{Part 4: Rectifiers}
This part of the lab focused on the rectifying of diodes. The circuit
shown in Fig. \ref{fig:rectifier_1} was configured.
\begin{figure}[ht]
  \centering
  \includegraphics[width=0.5\textwidth]{images/rectifier_circuit.png}
  \caption{Half-wave rectifier circuit.}
  \label{fig:rectifier_1}
\end{figure}
\par In a half-wave rectifier circuit, only the signal in the positive domain of
the wave is allowed to pass. In the experiment, the input signal is 60 Hz sine wave
with 5 $V$ amplitude. The simulation result is shown below in Fig. \ref{fig:rectifier_2}.
\begin{figure}[h]
  \centering
  \includegraphics[width=0.4\textwidth]{images/halfwave.png}
  \caption{Simulation of the half-wave-rectifier}
  \label{fig:rectifier_2}
\end{figure}
\par It can be observed that only half the wave is going through the load and the diode
triggers the voltage drop.
\newpage
\par For the circuit with capacitor in Fig. \ref{fig:rectifier_3}, a 1 $\mu$F
capacitor was placed in parallel with the load resistor.
\begin{figure}[h]
  \centering
  \includegraphics[width=0.5\textwidth]{images/rectifier_circuit_2.png}
  \caption{Circuit diagram of the rectifier with a 1 $\mu$F capacitor.}
  \label{fig:rectifier_3}
\end{figure}
\par The simulation result in Fig. \ref{fig:rectifier_4} shows that since the
capacitance is too small, it discharges very quickly.
\begin{figure}[h]
  \centering
  \includegraphics[width=0.35\textwidth]{images/halfwave_1miuf.png}
  \caption{Simulation of the rectifier with a 1 $\mu$F capacitor.}
  \label{fig:rectifier_4}
\end{figure}
\par When the capacitance is increased to 100 $\mu$F, which decreases the time
constant and increases the discharging time, the output amplitude shows the appearance
similar to a DC voltage. Fig. \ref{fig:rectifier_5} shows such behavior in the simulation.
\begin{figure}[h]
  \centering
  \includegraphics[width=0.35\textwidth]{images/halfwave_100miuf.png}
  \caption{Simulation of the rectifier with a 100 $\mu$F capacitor.}
  \label{fig:rectifier_5}
\end{figure}
\clearpage
\par For the half-rectifier with 100 $\mu$F capacitor, the peak amplitude of
the output function decreases to around 0.46 $V$. Given by $V_{p-p}$ = $\frac {1}{f \cdot C}$
, using this formula, we obtained a ripple frequency of 240 kHz when a 1 $\mu$F
capacitor is used and 22 kHz when a 100 $\mu$F capacitor is applied.
\subsection{Part 5: Voltage Regulation Using Zener Diode}
\begin{figure}[!hb]
  \centering
  \includegraphics[width=0.5\textwidth]{images/5_1.png}
  \caption{circuit diagram of the rectifier with a voltage divider at its output.}
  \label{fig:5.1}
\end{figure}
By using the basic format structure of the circuit from the previous section,
we replaced the 1 k$\Omega$ resistor to $R_{1}$ and attaching an addition
$R_{2}$ in series with $R_{1}$ to form a voltage divider (see Fig. \ref{fig:5.1}).
We constructed $R_{1}$ as two 100 $\Omega$ resistors in series with two 100
$\Omega$ resistor in parallel which would have an equivalent resistance of
250 $\Omega$, on the other hand we used two 1 k$\Omega$ resistors in parallel
and in series with Two 100 $\Omega$ in series which is in series of two 100
$\Omega$ resistor in parallel, in such a way we have $R_{2}$ = 750 $\Omega$.
\par The output voltage is measured in Fig. \ref{fig:5.2}.
\begin{figure}[!hb]
  \centering
  \includegraphics[width=0.25\textwidth]{images/5_2.png}
  \caption{output voltage with voltage divider (input is grey and output is in black)}
  \label{fig:5.2}
\end{figure}
\par The average output voltage is at 3.3V.
The ripple is similar to the one in part IV, however, its amplitude has been divided,
which makes this approach to have the smallest AC ripple.
\par We modified the circuit furthermore, by adding a zener diode in
series with $R_{2}$ and replace the oscilloscope to the node where the Zener diode
and $R_{2}$ attached as shown in Fig. \ref{fig:5.3}.
\begin{figure}[!hb]
  \centering
  \includegraphics[width=0.25\textwidth]{images/5_3.png}
  \caption{circuit diagram of the rectifier with a regulating Zener diode}
  \label{fig:5.3}
\end{figure}
\par The output measurement and the input measurement is as shown below in Fig. \ref{fig:5.4}.
\begin{figure}[ht]
  \centering
  \includegraphics[width=0.25\textwidth]{images/5_4.png}
  \caption{output voltage signal with a regulating zener diode}
  \label{fig:5.4}
\end{figure}
\par The average output signal is at around 2.3 $V$, however the signal itself
is smooth and have now visually detectable ripples. This circuit obviously
provides the least AC ripple among every other method we have tried previously
(including the resistor divider format).
\subsection{Part 6: Limiter Circuit Using Diodes}
\begin{figure}[ht]
  \centering
  \includegraphics[width=0.25\textwidth]{images/6_1.png}
  \caption{circuit diagram of the voltage limiter with 1.5 $V$ cell attached}
  \label{fig:6.1}
\end{figure}
We constructed the circuit as in Fig. \ref{fig:6.1} suggested, in which
this circuit is induced by using a signal with $V_{p-p} = 5 V$ and 0.5 $V$ DC offset.
The $V_{x}$ and $V_{y}$ is measured by using oscilloscope A and B. The result is
as shown in Fig. \ref{fig:6.2} below where $V_{x}$ is in grey and $V_{y}$ is in black.
\begin{figure}[ht]
  \centering
  \includegraphics[width=0.25\textwidth]{images/6_2.png}
  \caption{$V_{x}$ and $V_{y}$ signal measured with 1.5 $V$ cell battery in the circuit}
  \label{fig:6.2}
\end{figure}
\par We can see that after the as the input is passing 2 $V$, the battery-diode arrangement will turn on.
\par Now we remove the 1.5 $V$ battery pack and replacing it with an opposite
headed diode in parallel with existing diode in the circuit as shown in  Fig. \ref{fig:6.3}.
\begin{figure}[ht]
  \centering
  \includegraphics[width=0.25\textwidth]{images/6_3.png}
  \caption{circuit diagram with reverse direction diodes}
  \label{fig:6.3}
\end{figure}
\par We re-adjust the input signal by using the function generator to generate a
4 $V$ peak to peak and 0 $V$ DC offset signal. The measured $V_{x}$ and $V_{y}$
is as shown below in Fig. \ref{fig:6.4}.
\begin{figure}[ht]
  \centering
  \includegraphics[width=0.25\textwidth]{images/6_4.png}
  \caption{$V_{x}$ and $V_{y}$ signal with reverse direction diodes}
  \label{fig:6.4}
\end{figure}
\par This kind of circuit arrangement can be used in noise cancellation and
small signal applications, since the reversed direction diode can limited voltage
signal in both too high and too low, because if the nature of the diode too high
reversed bias would cause reverse breakdown, thus this kind of circuit does not
capable high voltage application.
\clearpage
\section{CONCLUSIONS}
\par To conclude, in this lab, various Op-Amp circuits were built and their behaviors were observed.
We saw in this lab that an op-amp contrary to its ideal model has an offset voltage,
and small currents entering its inputs. So we found experimentally that op-amps have biased input currents and offset currents.
\par In this lab, the experiments also showed that the gain of amplifiers is dependent on frequency. It was also observed that if the input is frequency is too high, the output will be disturbed and distorted.

% DO NOT NEED FROM HERE
%  \addtolength{\textheight}{-12cm}
% This command serves to balance the column lengths
% on the last page of the document manually. It shortens
% the textheight of the last page by a suitable amount.
% This command does not take effect until the next page
% so it should come on the page before the last. Make
% sure that you do not shorten the textheight too much.

%%%%%%%%%%%%%%%%%%%%%%%%%%%%%%%%%%%%%%%%%%%%%%%%%%%%%%%%%%%%%%%%%%%%%%%%%%%%%%%%



%%%%%%%%%%%%%%%%%%%%%%%%%%%%%%%%%%%%%%%%%%%%%%%%%%%%%%%%%%%%%%%%%%%%%%%%%%%%%%%%



%%%%%%%%%%%%%%%%%%%%%%%%%%%%%%%%%%%%%%%%%%%%%%%%%%%%%%%%%%%%%%%%%%%%%%%%%%%%%%%%

%    \section*{APPENDIX}
%
%    Appendixes should appear before the acknowledgment.
%
%    \section*{ACKNOWLEDGMENT}
%
%    The preferred spelling of the word ``acknowledgment'' in America is without an ``e'' after the ``g''. Avoid the stilted expression, ``One of us (R. B. G.) thanks . . .'' Instead, try ``R. B. G. thanks''. Put sponsor acknowledgments in the unnumbered footnote on the first page.
%
%
%
%    %%%%%%%%%%%%%%%%%%%%%%%%%%%%%%%%%%%%%%%%%%%%%%%%%%%%%%%%%%%%%%%%%%%%%%%%%%%%%%%%
%
%    References are important to the reader; therefore, each citation must be complete and correct. If at all possible, references should be commonly available publications.


%    \begin{thebibliography}{99}
%
%        \bibitem{c1} G. O. Young, ``Synthetic structure of industrial plastics (Book style with paper title and editor),'' in Plastics, 2nd ed. vol. 3, J. Peters, Ed. New York: McGraw-Hill, 1964, pp. 15--64.
%        \bibitem{c2} W.-K. Chen, Linear Networks and Systems (Book style). Belmont, CA: Wadsworth, 1993, pp. 123--135.
%        \bibitem{c3} H. Poor, An Introduction to Signal Detection and Estimation. New York: Springer-Verlag, 1985, ch. 4.
%        \bibitem{c4} B. Smith, ``An approach to graphs of linear forms (Unpublished work style),'' unpublished.
%        \bibitem{c5} E. H. Miller, ``A note on reflector arrays (Periodical styleÑAccepted for publication),'' IEEE Trans. Antennas Propagat., to be publised.
%        \bibitem{c6} J. Wang, ``Fundamentals of erbium-doped fiber amplifiers arrays (Periodical styleÑSubmitted for publication),'' IEEE J. Quantum Electron., submitted for publication.
%        \bibitem{c7} C. J. Kaufman, Rocky Mountain Research Lab., Boulder, CO, private communication, May 1995.
%        \bibitem{c8} Y. Yorozu, M. Hirano, K. Oka, and Y. Tagawa, ``Electron spectroscopy studies on magneto-optical media and plastic substrate interfaces(Translation Journals style),'' IEEE Transl. J. Magn.Jpn., vol. 2, Aug. 1987, pp. 740--741 [Dig. 9th Annu. Conf. Magnetics Japan, 1982, p. 301].
%        \bibitem{c9} M. Young, The Techincal Writers Handbook. Mill Valley, CA: University Science, 1989.
%        \bibitem{c10} J. U. Duncombe, ``Infrared navigationÑPart I: An assessment of feasibility (Periodical style),'' IEEE Trans. Electron Devices, vol. ED-11, pp. 34--39, Jan. 1959.
%        \bibitem{c11} S. Chen, B. Mulgrew, and P. M. Grant, ``A clustering technique for digital communications channel equalization using radial basis function networks,'' IEEE Trans. Neural Networks, vol. 4, pp. 570--578, July 1993.
%        \bibitem{c12} R. W. Lucky, ``Automatic equalization for digital communication,'' Bell Syst. Tech. J., vol. 44, no. 4, pp. 547--588, Apr. 1965.
%        \bibitem{c13} S. P. Bingulac, ``On the compatibility of adaptive controllers (Published Conference Proceedings style),'' in Proc. 4th Annu. Allerton Conf. Circuits and Systems Theory, New York, 1994, pp. 8--16.
%        \bibitem{c14} G. R. Faulhaber, ``Design of service systems with priority reservation,'' in Conf. Rec. 1995 IEEE Int. Conf. Communications, pp. 3--8.
%        \bibitem{c15} W. D. Doyle, ``Magnetization reversal in films with biaxial anisotropy,'' in 1987 Proc. INTERMAG Conf., pp. 2.2-1--2.2-6.
%        \bibitem{c16} G. W. Juette and L. E. Zeffanella, ``Radio noise currents n short sections on bundle conductors (Presented Conference Paper style),'' presented at the IEEE Summer power Meeting, Dallas, TX, June 22--27, 1990, Paper 90 SM 690-0 PWRS.
%        \bibitem{c17} J. G. Kreifeldt, ``An analysis of surface-detected EMG as an amplitude-modulated noise,'' presented at the 1989 Int. Conf. Medicine and Biological Engineering, Chicago, IL.
%        \bibitem{c18} J. Williams, ``Narrow-band analyzer (Thesis or Dissertation style),'' Ph.D. dissertation, Dept. Elect. Eng., Harvard Univ., Cambridge, MA, 1993.
%        \bibitem{c19} N. Kawasaki, ``Parametric study of thermal and chemical nonequilibrium nozzle flow,'' M.S. thesis, Dept. Electron. Eng., Osaka Univ., Osaka, Japan, 1993.
%        \bibitem{c20} J. P. Wilkinson, ``Nonlinear resonant circuit devices (Patent style),'' U.S. Patent 3 624 12, July 16, 1990.
%
%
%    \end{thebibliography}


\end{document}
