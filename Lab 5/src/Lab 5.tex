\documentclass[journal]{IEEEtran}
\usepackage[pdftex]{graphicx}
\usepackage[cmex10]{amsmath}
\usepackage{array}
\usepackage{mdwmath}
\usepackage{mdwtab}
\usepackage{eqparbox}
\usepackage{url}
\usepackage{pdfpages}
\usepackage{siunitx}

\begin{document}
\begin{titlepage}
\includepdf[page=-]{Cover.pdf}
\end{titlepage}
\title{Design of a BJT Amplifier}
\author{Chang Zhou, Siyu Wang, Haoran Du
\thanks{This work is the report of the laboratory section of course ECSE 331 offered at McGill University.}
\thanks{C. Zhou is with the Department of Electrical and Computer
Engineering, Faculty of Engineering, McGill University, H3A 0E9 Montreal, QC, Canada
(email: chang.zhou2@mail.mcgill.ca).}
\thanks{S. Wang is with the Department of Electrical and Computer
Engineering, Faculty of Engineering, McGill University, H3A 0E9 Montreal, QC, Canada
(email: siyu.wang5@mail.mcgill.ca).}
\thanks{H. Du is with the Department of Electrical and Computer
Engineering, Faculty of Engineering, McGill University, H3A 0E9 Montreal, QC, Canada
(email: haoran.du@mail.mcgill.ca).}}

\maketitle

\begin{abstract}
The purpose of laboratory experiment is to design a BJT amplifier using the 2N2222A npn transistor that we studied in
Lab 4 using the NI Elvis-II test instrument. In the laboratory, an amplifier with a gain of 50 \si{\volt}/\si{\volt} was built. Then in the second part, its gain at various temperatures were measured and compared. Experimental results shows that as the temperature of the transistors increases, the gain also increases.
\end{abstract}

\begin{IEEEkeywords}
  BJT, transistors, Amplifier
\end{IEEEkeywords}

\section{Introduction}

\IEEEPARstart {T}{he} goal of this laboratory was to build, test, and explore the behavior of the BJT amplifier using the NI Elvis-II test instrument. The performance of the amplifier was tested under the required constrains of bandwidth and RMS level. The effect of temperature on the operation of the transistors was also tested and found.
\section{Experiments Procedures and Result}
\subsection{BJT Amplifier Circuit Design}
\subsection{BJT Amplifier Performance Testing}
\subsection{Temperature Effects of BJT Amplifier}


\section{Conclusion}
\par To conclude, this lab was very helpful for using a BJT transistor. We built an amplifier that produced a gain of around 50 \si{\volt}/\si{\volt}for a frequency range of over \SI{100}{\kilo\hertz}.
The behavior of the amplifier was tested at various temperatures, the gain was found using the Bode plot at room temperature (\SI{27}{\celsius}), \SIlist{-10;-19;63}{\celsius}. The differences in the gain were very small, but overall satisfying. Our gain increased slightly as we decreased the temperature of the BJT. This could have been due to the fact that, at lower temperature, there are less mobile charge carriers.
\end{document}
