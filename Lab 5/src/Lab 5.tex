\documentclass[journal]{IEEEtran}
\usepackage[pdftex]{graphicx}
\usepackage[cmex10]{amsmath}
\usepackage{array}
\usepackage{mdwmath}
\usepackage{mdwtab}
\usepackage{eqparbox}
\usepackage{url}
\usepackage{pdfpages}
\usepackage{siunitx}
\usepackage{float}

\begin{document}
\begin{titlepage}
\includepdf[page=-]{Cover.pdf}
\end{titlepage}
\title{Design of a BJT Amplifier}
\author{Chang Zhou, Siyu Wang, Haoran Du
\thanks{This work is the report of the laboratory section of course ECSE 331 offered at McGill University.}
\thanks{C. Zhou is with the Department of Electrical and Computer
Engineering, Faculty of Engineering, McGill University, H3A 0E9 Montreal, QC, Canada
(email: chang.zhou2@mail.mcgill.ca).}
\thanks{S. Wang is with the Department of Electrical and Computer
Engineering, Faculty of Engineering, McGill University, H3A 0E9 Montreal, QC, Canada
(email: siyu.wang5@mail.mcgill.ca).}
\thanks{H. Du is with the Department of Electrical and Computer
Engineering, Faculty of Engineering, McGill University, H3A 0E9 Montreal, QC, Canada
(email: haoran.du@mail.mcgill.ca).}}

\maketitle

\begin{abstract}
The purpose of laboratory experiment is to design a BJT amplifier using the 2N2222A npn transistor that we studied in
Lab 4 using the NI Elvis-II test instrument. In the laboratory, an amplifier with a gain of 50 \si{\volt}/\si{\volt} was built. Then in the second part, its gain at various temperatures were measured and compared. Experimental results shows that as the temperature of the transistors increases, the gain also increases.
\end{abstract}

\begin{IEEEkeywords}
  BJT, transistors, Amplifier
\end{IEEEkeywords}

\section{Introduction}

\IEEEPARstart {T}{he} goal of this laboratory was to build, test, and explore the behavior of the BJT amplifier using the NI Elvis-II test instrument. The performance of the amplifier was tested under the required constrains of bandwidth and RMS level. The effect of temperature on the operation of the transistors was also tested and found.
\section{Experiments Procedures and Result}
\subsection{BJT Amplifier Circuit Design} \label{1.1}
\begin{figure}[h]
  \centering
  \includegraphics[width=0.45\textwidth]{images/circuit.png}
  \caption{Design of BJT amplifier.}
  \label{fig-1-1}
\end{figure}
\par The BJT amplifier design as shown in Fig. \ref{fig-1-1} was provided in the
sample circuit, where $R_{1}$ to $R_{6}$ are \SIlist{3;1.3;66.5;33;10;0.024}{\kilo\ohm}
respectively, and all three capacitors are \SI{47}{\micro\F}. Ideally, the gain should equal to $\frac{R_{1} || R_{5}}{R_{6}}$ if the capacitor was large enough; however, when we constructed the given circuit, the gain we found was only around 38 \si{\volt}/\si{\volt}. This was because the capacitor was not ideal.
As a result, we wanted to increase the gain. There are three ways to increase the gain:
to increase $R_{1}$, to increase $R_{5}$, or to decrease $R_{6}$. However, during the experiment, we found that if we were to increase $R_{1}$ slightly, the output signal we have would have a cutoff region on its sine wave.
Therefore, we decided to increase $R_{5}$ to \SI{1}{\Mohm} and decrease $R_{6}$ to \SI{14}{\ohm}.

\subsection{BJT Amplifier Performance Testing}
\par After setting up the circuit as shown in  Fig. \ref{fig-1-1} with the element we have mentioned in section \ref{1.1}, we modified our input as $V_{pp} =$ \SI{0.66}{\volt}; \SI{1}{\kilo\hertz} sine wave; \SI{0}{\volt} offset. The measured result from the two oscilloscope shows (in Fig. \ref{fig-2-1} (where black is the input signal, and grey line is the output signal) that this circuit has a voltage gain of 47.86 \si{\volt}/\si{\volt}, which is in the desired range.
\begin{figure}[h]
  \centering
  \includegraphics[width=0.45\textwidth]{images/2-1.png}
  \caption{Measured Output voltage and input voltage at room temperature (\SI{27}{\celsius}) with input voltage at \SI{1}{\kilo\hertz}}
  \label{fig-2-1}
\end{figure}
\clearpage
\par Next, without modifying any circuit elements, we changed our the frequency of our
input voltage to 11 KHz. As shown in Fig. \ref{fig-2-2}, the gain is obtained as 55 \si{\volt}/\si{\volt}, which is
acceptably desirable.
\begin{figure}[h]
\centering
\includegraphics[width=0.45\textwidth]{images/2-2.png}
  \caption{Measured Output voltage and input voltage at room temperature (\SI{27}{\celsius}) with input voltage at \SI{11}{\kilo\hertz}}
\label{fig-2-2}
\end{figure}
\par Finally, the frequency and of the input was inverted back to \SI{1}{\kilo\hertz} but added with \SI{1}{\volt} offset.
The measured gain under such circumstane is 45 \si{\volt}/\si{\volt} as shown below in Fig. \ref{fig-2-3}.
\begin{figure}[h]
\centering
\includegraphics[width=0.45\textwidth]{images/2-3.png}
  \caption{Measured Output voltage and input voltage at room temperature (\SI{27}{\celsius}) with input voltage at \SI{11}{\kilo\hertz}}
\label{fig-2-3}
\end{figure}
\subsection{Temperature Effects of BJT Amplifier}
\begin{figure}[H]
  \centering
  \includegraphics[width=0.45\textwidth]{images/27degree.png}
  \caption{$V_{in}$ and $V_{out}$ of the circuit at room temperature (\SI{27}{\celsius}).}
  \label{fig-3-1}
\end{figure}
As seen in Fig. \ref{fig-3-1}, the gain of the BJT amplifier at at room temperature (\SI{27}{\celsius}) is
\begin{equation*}
  \begin{split}
    G&=\frac{V_{out}}{V_{in}}\\
    &=\frac{\SI{4.864}{\volt}}{\SI{83.88}{\milli\volt}}\\
    &=58.0 \si{\volt}/\si{\volt}
  \end{split}
\end{equation*}
\begin{figure}[h]
  \centering
  \includegraphics[width=0.45\textwidth]{images/63degree.png}
  \caption{$V_{in}$ and $V_{out}$ of the circuit at \SI{63}{\celsius}.}
  \label{fig-3-2}
\end{figure}
When the temperature is increased to \SI{63}{\celsius}, as seen in Fig. \ref{fig-3-2}, we have the gain
\begin{equation*}
  \begin{split}
    G&=\frac{V_{out}}{V_{in}}\\
    &=\frac{\SI{3.690}{\volt}}{\SI{100.66}{\milli\volt}}\\
    &=36.9 \si{\volt}/\si{\volt}
  \end{split}
\end{equation*}
Thus, it is observed that the gain of the BJT amplifier decreases as its temperature increases.
Similarly, we conducted similar operations to observe its temperature characteristics when the
circuit is brought down from room temperature.
\begin{figure}[h]
  \includegraphics[width=0.45\textwidth]{images/-10degree.png}
  \caption{BJT amplifier at \SI{-10}{\celsius}.}
  \label{fig-3-3}%
\end{figure}
Similarly, as given in Fig. \ref{fig-3-3}, we could calculate that
the gain of the amplifier at \SI{-10}{\celsius} is 55.64 \si{\volt}/\si{\volt}. It
can be observed that the gain has a small decrease as the temperature is brought down,
which is surprisingly not pertaining with the theory.

\section{Conclusion}
\par To conclude, this lab was very helpful for using a BJT transistor. We built an amplifier that produced a gain of around 55 \si{\volt}/\si{\volt}for a frequency range of over \SI{100}{\kilo\hertz}.
The behavior of the amplifier was tested at various temperatures, the gain was found using the Bode plot at room temperature (\SI{27}{\celsius}), \SIlist{-10;63}{\celsius}. The differences in the gain were very small, but overall satisfying. Our gain increased slightly as we decreased the temperature of the BJT. This could have been due to the fact that, at lower temperature, there are less mobile charge carriers.
\end{document}
